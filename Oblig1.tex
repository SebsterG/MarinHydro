\documentclass[a4paper,norsk]{article}
\usepackage{preamble}


\begin{document}

\title{Marine Hydrodynamics \\ Assignment 1}
\author{Sebastian Gjertsen}

\maketitle

\begin{abstract}
For specified (see below) two-dimensional geometries, assuming potential theory in unbounded fluid, and use of Green's second identity, calculate the velocity potential along the body and the added mass forces, for a circle, an ellipse, a square and a rectangle, moving laterally, and with rotation. Find also the cross coupling added mass coefficients. For the circle, the reference solution is: $ \phi =-a^2x/(^2)$ where a denotes the cylinder radius, $r^2=x^2+y^2$.
\end{abstract}


\section{Teori}
In this assignment we use the method of panels. By splitting the geometry into N equal parts, and assuming that $\phi , \frac{\partial \phi }{\partial n}$ is constant every segment.
From Newman chapter 4 we have (79):
\[   \int \phi \frac{\partial G }{\partial n} - G\frac{\partial \phi }{\partial n} = -\pi \phi(x,y) \]
where $G = ln (r)$, a source in 2D
$$ -\pi \phi(x_0) + \int_C \phi \frac{\partial }{\partial n} ln(r) dl = \int ln(r) \frac{\partial \phi}{\partial n} dl       $$
where $x_0$ states a point on our geometry.\\
\newline
We got a trick from the lectures, turning it into:
\[ -\pi \phi(x_0) + \sum_{n=1}^N \phi(x_n) (\theta_a - \theta_b)  = \int ln(r)\frac{\partial \phi }{\partial n} dl    \]
$$
\begin{pmatrix}
  -\pi & (\theta_a - \theta_b)_0  & (\theta_a - \theta_b)_1 & ...\\
  (\theta_a - \theta_b)_N  & -\pi & (\theta_a - \theta_b)_0  & ...\\
  (\theta_a - \theta_b)_{N-1}  & (\theta_a - \theta_b)_N  & -\pi & ...\\
  ... & ... & ....& ...  \\
 \end{pmatrix}
 \begin{pmatrix}
\phi_0  \\
\phi_1 \\
\phi_2 \\ 
  ...  \\
  \phi_N
 \end{pmatrix}
 =
 \begin{pmatrix}
\int ln(r_0)\frac{\partial \phi }{\partial n} dl   \\
\int ln(r_1)\frac{\partial \phi }{\partial n} dl  \\
\int ln(r_2)\frac{\partial \phi }{\partial n} dl  \\ 
  ...  \\
  \end{pmatrix}
  $$
To calculate the $\phi$ values we created a fictional point between the $x_N$ points, where we stand in this "ghost" point and calculate the $\Delta \theta$ to the $x_N$ values. This is done since we have assumed $\phi$ to constant.
To evaluate the integrals we used the trapezoidal method. Where we again stand in the "ghost" point and calculate between the 





\section{Conclusion}



\end{document}